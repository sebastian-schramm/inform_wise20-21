\section{Reflexion}

\begin{itemize}
	\item Was waren Ihre zwei größten Herausforderungen in diesem Modul?
	\subitem Einmal die Aufsätze zu schreiben, da es mir bei einigen Themen schwer viel, aufgrund der menge an wichtigen Informationen. Diese aber nur in ein Paar setzen wiederzugeben.
	\subitem Und andererseits haben wir eine Aufgabe zur Turing Maschine bekommen. Den egal was ich da
	eingestellt hatte, sie wollte einfach nicht so wie ich es gerne hätte. 
	
	\item Was zeigt das Portfolio von Ihnen und Ihrer Arbeit?
	\subitem Was ich in diesem Modul gelernt habe.
	
	\item Wo sehen Sie noch Fehlstellen und Lernmöglichkeiten ihrerseits?
	\subitem Mich mehr mit Automaten beschäftigen.
	
	\item Auf welche Bereiche ließe sich das von Ihnen Gelernte übertragen?
	\subitem Ein großer Punkt werden vermutlich die UML Diagramme sein. Da diese essentiell für die Softwareentwicklung sind. 
\end{itemize}

1.Zu Beginn des Moduls haben Sie Erwartungen gehabt, was Sie wahrscheinlich innerhalb des Moduls lernen werden und wo Sie diese Erfahrungen später anwenden können. Erläutern Sie, inwiefern diese Aussagen zutreffen und inwiefern nicht. Wie kommen ggf. Unterschiede zustande?
2.Wie sehen Sie aus heutiger Sicht, wo Sie die Lernerfahrungen dieses Moduls im Arbeitsleben anwenden können?
%3.Was waren Ihre zwei größten Herausforderungen in diesem Modul?
4.Wie sind Sie mit diesen Herausforderungen umgegangen?
%5.Was zeigt das Portfolio von Ihnen und Ihrer Arbeit?
%6.Wo sehen Sie noch Fehlstellen und Lernmöglichkeiten ihrerseits?
%7.Auf welche Bereiche ließe sich das von Ihnen Gelernte übertragen?