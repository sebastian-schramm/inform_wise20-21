\section{Reflexion}

\begin{itemize}
	\item Was waren Ihre zwei größten Herausforderungen in diesem Modul?
	\subitem Zum einen, die Umstellung, dass wir uns das Wissen selber beigringen mussten.
	Obwohl uns schon sehr viel beigebracht wurde, mussten wir für viele aufgaben uns Bücher zu
	hand nehmen oder im Internet recherchieren.
	\subitem Und zum anderen, dass wir keine Präsents Vorlesung hatten. Dadurch fehlte der
	Kontakt zu anderen mit studierenden. Wodurch man sich nicht wirklich mit anderen
	studierenden austauschen konnte.
	
	\item Wie sind Sie mit diesen Herausforderungen umgegangen?
	\subitem Nach kurzerzeit hatten wir neben der Whatsapp Gruppe noch einen Discord server,
	wo sich die studierenden treffen können. So haben sich auch ein paar Lergruppen gebildet.
	
	\item Wo sehen Sie noch Fehlstellen und Lernmöglichkeiten ihrerseits?
	\subitem Mich mehr mit Automaten beschäftigen.
	
	\item Auf welche Bereiche ließe sich das von Ihnen Gelernte übertragen?
	\subitem Ein großer Punkt werden vermutlich die UML Diagramme sein. Da diese essentiell für die Softwareentwicklung sind. 
\end{itemize}

1.Zu Beginn des Moduls haben Sie Erwartungen gehabt, was Sie wahrscheinlich innerhalb des Moduls lernen werden und wo Sie diese Erfahrungen später anwenden können. Erläutern Sie, inwiefern diese Aussagen zutreffen und inwiefern nicht. Wie kommen ggf. Unterschiede zustande?
2.Wie sehen Sie aus heutiger Sicht, wo Sie die Lernerfahrungen dieses Moduls im Arbeitsleben anwenden können?
%3.Was waren Ihre zwei größten Herausforderungen in diesem Modul?
4.Wie sind Sie mit diesen Herausforderungen umgegangen?
%5.Was zeigt das Portfolio von Ihnen und Ihrer Arbeit?
%6.Wo sehen Sie noch Fehlstellen und Lernmöglichkeiten ihrerseits?
%7.Auf welche Bereiche ließe sich das von Ihnen Gelernte übertragen?