\section{Automaten}
Schreiben Sie einen Aufsatz zum Thema „Automaten“. Dieser Aufsatz sollte maximal 3000 Zeichen lang sein (ca. zwei DIN A4 Seiten).

Automaten
Ein Automat ist ein Bestandteil der Informatik. Es is egal, ob ein System sinnvoll oder gar möglich ist zu bauen. Mithilfe eines Automaten kann das Verhalten leichter verstanden werden. Im Prinzip ist das Verhalten eines Automaten immer gleich. „Der Automat befindet sich in einem bestimmten Zustand. Jedes Mal, wenn ein Eingabezeichen eintrifft, kann sich abhängig vom Eingabezeichen und dem gegenwärtigen Zustand ein neuer Zustand, der Folgezustand, einstellen.“ (Automat, 2021)
Es wird unterschieden zwischen einen Deterministischen und nichtdeterministischen Automaten. Bei einem  Deterministischen Automaten treten nur definierte und reproduzierbare Zustände auf. Das heißt, wenn ich ein Wort einlese, werden die selben Zustände durchlaufen und ich bekomme so am ende die selbe Ausgabe. „Der Begriff Determinismus ist vom Begriff Determiniertheit zu unterscheiden: Ein deterministischer Algorithmus ist immer determiniert, d. h., er liefert bei gleicher Eingabe immer die gleiche Ausgabe“. (Determinismus, 2021) Andersherum gilt dies nicht. Zum Beispiel ist der Quicksort nicht-deterministisch, weil dessen Zwischenergebnisse sich stets unterscheiden, jedoch ist das Ergebnis nach der Terminierung immer identisch. (vgl, Determinismus, 2021) 
Während hingegen bei einen nichtdeterministischen Automaten bei gleicher Eingabe mehrere mögliche Durchlaufmöglichkeiten existieren. Im Allgemeinen sind diese Modelle nur theoretisch und praktisch nicht realisierbar. In der theoretischen Informatik dienen sie dazu da, um herauszufinden ob eine bestimmtes Problem mit einem nichtdeterministischen Algorithmus angegeben werden kann. Da sich ein Problem damit leichter lösen lässt. (vgl, Nichtdeterminismus, 2021) 
Unteranderem gibt es auch Automaten für die Ein- und Ausgabe. Hierbei wird unterschieden zwischen einen Akzeptor, welcher ein spezieller endlicher Automat ist. Dieser erzeugt keine Ausgaben, kann aber Wörter einlesen. Diese werden Zeichen für Zeichen eingelesen, wobei diese im selben Zustand bleiben können, oder in einen neuen übergehen. Zusätzlich besitzen sie einen Startzustand und ein Endzustand und nur dann, wenn der Finalzustand im Endzustand terminiert, gilt die Eingabe als akzeptiert. Ein Transduktor zeichnet sich dadurch aus, dass er keine Ausgabe erzeugt. Im Gegenteil, der Transduktor entwickelt aus einer vordefinierten Quellspache ein Wort. 
Bei einer ein oder Ausgabe hat meinen ein Akzeptor oder einen Transduktor. Ein Akzeptor besitzt einen Startzustand und einen Endzustand. Hier wird eine Wort eingelesen und Zeichen für Zeichen verarbeitet. Wenn der Automat am ende im Endzustand aufhört, wird die Eingabe akzeptiert. Im Gegensatz zum Akzeptor, erzeugt der Transduktor eine Ausgabe. Hier werden entweder jedem oder Paarweise der Zustand vom Eingabezeichen zu einen Ausgabezeichen zugeordnet. 
Aber welche Automaten werden in der Praxis bei der Programmierung verwendet? Dafür kommen Endliche Automaten und Kellerautomaten in frage, da diese komplexe Probleme übersichtlich lösen lassen können. (vgl, Automat, 2021) 

\begin{lstlisting}
	https://de.wikipedia.org/wiki/Automat_(Informatik)
	https://de.wikipedia.org/wiki/Determinismus_(Algorithmus)
	https://de.wikipedia.org/wiki/Nichtdeterminismus
\end{lstlisting}